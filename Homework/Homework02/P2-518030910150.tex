\documentclass[12pt,a4paper]{article}
\usepackage{ctex}
\usepackage{amsmath,amscd,amsbsy,amssymb,latexsym,url,bm,amsthm}
\usepackage{epsfig,graphicx,subfigure}
\usepackage{enumerate}
\usepackage{wrapfig}
\usepackage{mathrsfs,euscript}
\usepackage[usenames]{xcolor}
\usepackage{hyperref}
\usepackage[vlined,ruled,linesnumbered]{algorithm2e}
\hypersetup{colorlinks=true,linkcolor=black}

\newtheorem{theorem}{Theorem}
\newtheorem{lemma}[theorem]{Lemma}
\newtheorem{proposition}[theorem]{Proposition}
\newtheorem{corollary}[theorem]{Corollary}
\newtheorem{exercise}{Exercise}
\newtheorem*{solution}{Solution}
\newtheorem{definition}{Definition}
\theoremstyle{definition}

\renewcommand{\thefootnote}{\fnsymbol{footnote}}

\newcommand{\postscript}[2]
 {\setlength{\epsfxsize}{#2\hsize}
  \centerline{\epsfbox{#1}}}

\renewcommand{\baselinestretch}{1.0}

\setlength{\oddsidemargin}{-0.365in}
\setlength{\evensidemargin}{-0.365in}
\setlength{\topmargin}{-0.3in}
\setlength{\headheight}{0in}
\setlength{\headsep}{0in}
\setlength{\textheight}{10.1in}
\setlength{\textwidth}{7in}
\makeatletter \renewenvironment{proof}[1][Proof] {\par\pushQED{\qed}\normalfont\topsep6\p@\@plus6\p@\relax\trivlist\item[\hskip\labelsep\bfseries#1\@addpunct{.}]\ignorespaces}{\popQED\endtrivlist\@endpefalse} \makeatother
\makeatletter
\renewenvironment{solution}[1][Solution] {\par\pushQED{\qed}\normalfont\topsep6\p@\@plus6\p@\relax\trivlist\item[\hskip\labelsep\bfseries#1\@addpunct{.}]\ignorespaces}{\popQED\endtrivlist\@endpefalse} \makeatother

\begin{document}
\noindent

%========================================================================
\noindent\framebox[\linewidth]{\shortstack[c]{
\Large{\textbf{Homework 02}}\vspace{1mm}\\
CS499-Mathematical Foundations of Computer Science, Jie Li, Spring 2020.}}
\begin{center}
\footnotesize{\color{blue}Name: ������ (Hongjie Fang)  \quad Student ID: 518030910150 \quad Email: galaxies@sjtu.edu.cn}
\end{center}

\begin{enumerate}[1.]
  \item What does the following notation mean?
  \begin{displaymath}
  \sum_{k=4}^0 q_k
  \end{displaymath}

  \begin{solution}
    The meaning of the formula depends on the specific situation.
    \begin{itemize}
    \item It can have the same meaning as $\sum_{4 \leq k \leq 0} q_k$, so the value of this formula is $0$.
    \item It can have the same meaning as $\sum_{k \leq 0} q_k - \sum_{k \leq 4} q_k$. As a result, the value of this formula is $-q_1 - q_2 - q_3 - q_4$.
    \end{itemize}
  \end{solution}

  \item Simplify the expression $x \cdot ([x > 0] - [x < 0])$.

  \begin{solution}
    Let $f(x)$ be $x \cdot ([x > 0] - [x < 0])$. Then we are able to derive the following equation (Equation \eqref{eq2.1}).
    \begin{equation}
        f(x) = \left\{
        \begin{aligned}
        & x \cdot (1 - 0) = x & \quad (x > 0) \\
        & 0 \cdot (0 - 0) = 0 & \quad (x = 0) \\
        & x \cdot (0 - 1) = -x & \quad (x < 0) \\
        \end{aligned}
        \right\} = |x|
        \label{eq2.1}
    \end{equation}
    Thus, $x \cdot ([x > 0] - [x < 0])$ can be simplified to $|x|$.
  \end{solution}

  \item Demonstrate your understanding of $\Sigma$-notation by writing out the following sums in full.
  \begin{displaymath}
  \sum_{0\leq k\leq 5} a_k \quad \quad \text{and} \quad \quad \sum_{0 \leq k^2 \leq 5} a_{k^2}
  \end{displaymath}

  \begin{solution} The sums can be written as follows (Equation \eqref{eq3.1}).
    \begin{equation}
    \begin{aligned}
        \sum_{0 \leq k \leq 5} a_k = a_0 + a_1 + a_2 + a_3 + a_4 + a_5 \\
        \sum_{0 \leq k^2 \leq 5} a_{k^2} = \sum_{-2 \leq k \leq 2} a_{k^2} = a_0 + 2a_1 + 2a_4
    \end{aligned}
    \label{eq3.1}
    \end{equation}
  \end{solution}

  \item Express the following triple sum as a three-fold summation (with three $\Sigma$'s).
  \begin{displaymath}
  \sum_{1 \leq i < j < k \leq 4} a_{ijk}
  \end{displaymath}
  \begin{enumerate}[(a) \quad]
  \item summing first on $k$, then $j$, then $i$;
  \item summing first on $i$, then $j$, then $k$.
  \end{enumerate}
  Also write your triple sums out in full without the $\Sigma$-notation, using parentheses to show what is being added together first.

  \begin{solution} The sums can be written as follows (Equation \eqref{eq4.1} and Equation \eqref{eq4.2}).
    \begin{enumerate}[(a) \quad]
    \item
        \begin{equation}
        \sum_{1 \leq i < j < k \leq 4} a_{ijk} = \sum_{i=1}^2 \sum_{j=i+1}^3 \sum_{k=j+1}^4 a_{ijk} = ((a_{123} + a_{124}) + a_{134}) + a_{234}
        \label{eq4.1}
        \end{equation}
    \item
        \begin{equation}
        \sum_{1 \leq i < j < k \leq 4} a_{ijk} = \sum_{k=3}^4 \sum_{j=2}^{k-1} \sum_{i=1}^{j-1} a_{ijk} = a_{123} + (a_{124} + (a_{134} + a_{234}))
        \label{eq4.2}
        \end{equation}
    \end{enumerate}
  \end{solution}

  \item What's wrong with the following derivation?
  \begin{displaymath}
  \left(\sum_{j=1}^n a_j\right)\left(\sum_{k=1}^n \frac{1}{a_k}\right) = \sum_{j=1}^n \sum_{k=1}^n \frac{a_j}{a_k} = \sum_{k=1}^n \sum_{k=1}^n \frac{a_k}{a_k} = \sum_{k=1}^n n = n^2
  \end{displaymath}

  \begin{solution}
    We can't change the variable $j$'s name to $k$, because the formula already has a variable $k$ in the inner summation. Therefore, we can only derive the following result (Equation \eqref{eq5.1}).
    \begin{equation}
    \left(\sum_{j=1}^n a_j\right)\left(\sum_{k=1}^n \frac{1}{a_k}\right) = \sum_{j=1}^n \sum_{k=1}^n \frac{a_j}{a_k}
    \label{eq5.1}
    \end{equation}
  \end{solution}

  \item What is the value of $\sum_k [1\leq j \leq k \leq n]$, as a function of $j$ and $n$?

  \begin{solution}
    We define $f(j, n)$ as the summation above. Thus, we can derive the following answer (Equation \eqref{eq6.1}).
    \begin{equation}
        f(j,n) = \sum_k [1\leq j \leq k \leq n] = \left\{
        \begin{aligned}
        & 0 & \quad \quad (j < 1 \text{\ or \ } j > n) \\
        & \sum_{j \leq k \leq n} 1 = n - j + 1 & \quad \quad (1 \leq j \leq n)
        \end{aligned}
        \right.
        \label{eq6.1}
    \end{equation}
  \end{solution}

  \item Let $\nabla f(x) = f(x) - f(x-1)$. What is $\nabla(x^{\overline{m}})$?

  \begin{solution} $\nabla(x^{\overline{m}})$ can be calculated as follows (Equation \eqref{eq7.1}).
    \begin{equation}
    \begin{aligned}
    \nabla(x^{\overline{m}}) & = x^{\overline{m}} - (x-1)^{\overline{m}} \\
                             & = \prod_{i=0}^{m-1} (x+i) - \prod_{i=0}^{m-1} (x-1+i) \\
                             & = (x + m - 1 - (x - 1)) \prod_{i=0}^{m-2} (x+i) \\
                             & = m x^{\overline{m-1}}
    \end{aligned}
    \label{eq7.1}
    \end{equation}
  \end{solution}

  \item What is the value of $0^{\underline{m}}$, where $m$ is a given integer?

  \begin{solution}
  \begin{equation}
    0^{\underline{m}} = \left\{
    \begin{aligned}
        & \prod_{i=1-m}^0 i = 0 & \quad (m > 0) \\
        & 1 & \quad (m = 0) \\
        & \prod_{i=1}^{-m} \frac{1}{i} = \frac{1}{(-m) !} & \quad (m < 0)
    \end{aligned}
    \right. \quad \quad (m \in \mathbb{N})
    \label{eq8.1}
  \end{equation}
    We assume $0! = 1$, then we can simplify Equation \eqref{eq8.1} to the following equation (Equation \eqref{eq8.2})
  \begin{equation}
    0^{\underline{m}} = \left\{
    \begin{aligned}
        & 0 & \quad (m > 0) \\
        & \frac{1}{(-m) !} & \quad (m \leq 0)
    \end{aligned}
    \right. \quad \quad (m \in \mathbb{N})
    \label{eq8.2}
  \end{equation}
  \end{solution}

  \item What is the law of exponents for rising factorial powers, analogous to (2.52) in the textbook? Use this to define $x^{\overline{-n}}$.

  \begin{solution} The law of exponents for rising factorial powers is as follows.
    \begin{equation}
        x^{\overline{m+n}} = x^{\overline{m}}(x+m)^{\overline{n}}  \quad \quad (m, n \in \mathbb{Z})
        \label{eq9.1}
    \end{equation}

    Thus, we can define $x^{\overline{-n}}$ by applying $m = -n$ and $n = n$ in Equation \eqref{eq9.1}, and we can derive the following formulas.

    \begin{equation}
    \begin{aligned}
                        & \quad x^{\overline{-n}} (x-n)^{\overline{n}} = x^{\overline{-n + n}} = x^{\overline{0}} = 1 \\
        \Longrightarrow & \quad x^{\overline{-n}} = \frac{1}{(x-n)^{\overline{n}}} = \frac{1}{(x-1)^{\underline{n}}} \quad \quad (n \in \mathbb{N_+})
    \end{aligned}
    \label{eq9.2}
    \end{equation}

    With Equation \eqref{eq9.2}, we define $x^{\overline{-n}}$ as $\frac{1}{(x-1)^{\underline{n}}}$ when $n \in \mathbb{N_+}$.
  \end{solution}

  \item The text derives the following formula for the difference of a product:
  \begin{equation}
    \Delta(uv) = u\Delta{v} + Ev\Delta{u}
    \label{eq10.1}
  \end{equation}
  How can this formula be correct, when the left-hand side is symmetric with respect to $u$ and $v$ but the right-hand side is not?

  \begin{solution}
    The derivation of Equation \eqref{eq10.1} is on the textbook, so it is obviously correct. Actually, on one hand, we have the Equation \eqref{eq10.1}; on the other hand, we can have the following equation (Equation \eqref{eq10.2}).
    \begin{equation}
    \Delta(uv) = v\Delta{u} + Eu\Delta{v}
    \label{eq10.2}
    \end{equation}
    The two right-hand sides of Equation \eqref{eq10.1} and Equation \eqref{eq10.2} are symmetric.

    What's more, if $u$ and $v$ are differentiable functions, then we can substitute the notation $\Delta$ with the notation $\mathrm{D}$ by taking the limit as the delta number of independent variable goes to $0$. As a result, $Ev$ equals to $v$ under $\mathrm{D}$. So the equation becomes $\mathrm{D}(uv) = u\mathrm{D}u + v\mathrm{D}u$, which is symmetric on both sides.
  \end{solution}
\end{enumerate}
%========================================================================
\end{document}
